%%
%% This is file `sample-sigconf.tex',
%% generated with the docstrip utility.
%%
%% The original source files were:
%%
%% samples.dtx  (with options: `sigconf')
%% 
%% IMPORTANT NOTICE:
%% 
%% For the copyright see the source file.
%% 
%% Any modified versions of this file must be renamed
%% with new filenames distinct from sample-sigconf.tex.
%% 
%% For distribution of the original source see the terms
%% for copying and modification in the file samples.dtx.
%% 
%% This generated file may be distributed as long as the
%% original source files, as listed above, are part of the
%% same distribution. (The sources need not necessarily be
%% in the same archive or directory.)
%%
%%
%% Commands for TeXCount
%TC:macro \cite [option:text,text]
%TC:macro \citep [option:text,text]
%TC:macro \citet [option:text,text]
%TC:envir table 0 1
%TC:envir table* 0 1
%TC:envir tabular [ignore] word
%TC:envir displaymath 0 word
%TC:envir math 0 word
%TC:envir comment 0 0
%%
%%
%% The first command in your LaTeX source must be the \documentclass
%% command.
%%
%% For submission and review of your manuscript please change the
%% command to \documentclass[manuscript, screen, review]{acmart}.
%%
%% When submitting camera ready or to TAPS, please change the command
%% to \documentclass[sigconf]{acmart} or whichever template is required
%% for your publication.
%%
%%
\documentclass[sigconf]{acmart}

%%
%% \BibTeX command to typeset BibTeX logo in the docs
\AtBeginDocument{%
  \providecommand\BibTeX{{%
    Bib\TeX}}}

%% Rights management information.  This information is sent to you
%% when you complete the rights form.  These commands have SAMPLE
%% values in them; it is your responsibility as an author to replace
%% the commands and values with those provided to you when you
%% complete the rights form.
\setcopyright{acmcopyright}
\copyrightyear{2018}
\acmYear{2018}
\acmDOI{XXXXXXX.XXXXXXX}

%% These commands are for a PROCEEDINGS abstract or paper.
\acmConference[Conference acronym 'XX]{Make sure to enter the correct
  conference title from your rights confirmation emai}{June 03--05,
  2018}{Woodstock, NY}
%%
%%  Uncomment \acmBooktitle if the title of the proceedings is different
%%  from ``Proceedings of ...''!
%%
%%\acmBooktitle{Woodstock '18: ACM Symposium on Neural Gaze Detection,
%%  June 03--05, 2018, Woodstock, NY}
\acmPrice{15.00}
\acmISBN{978-1-4503-XXXX-X/18/06}


%%
%% Submission ID.
%% Use this when submitting an article to a sponsored event. You'll
%% receive a unique submission ID from the organizers
%% of the event, and this ID should be used as the parameter to this command.
%%\acmSubmissionID{123-A56-BU3}

%%
%% For managing citations, it is recommended to use bibliography
%% files in BibTeX format.
%%
%% You can then either use BibTeX with the ACM-Reference-Format style,
%% or BibLaTeX with the acmnumeric or acmauthoryear sytles, that include
%% support for advanced citation of software artefact from the
%% biblatex-software package, also separately available on CTAN.
%%
%% Look at the sample-*-biblatex.tex files for templates showcasing
%% the biblatex styles.
%%

%%
%% The majority of ACM publications use numbered citations and
%% references.  The command \citestyle{authoryear} switches to the
%% "author year" style.
%%
%% If you are preparing content for an event
%% sponsored by ACM SIGGRAPH, you must use the "author year" style of
%% citations and references.
%% Uncommenting
%% the next command will enable that style.
%%\citestyle{acmauthoryear}
%%
%% end of the preamble, start of the body of the document source.
\begin{document}

%%
%% The "title" command has an optional parameter,
%% allowing the author to define a "short title" to be used in page headers.
\title{Unveiling Simplicity in Complexity}

%%
%% The "author" command and its associated commands are used to define
%% the authors and their affiliations.
%% Of note is the shared affiliation of the first two authors, and the
%% "authornote" and "authornotemark" commands
%% used to denote shared contribution to the research.

\author{John Doe}
\authornote{Both authors contributed equally to this research.}
\email{johndoe@johndoe.edu}

\author{John Doe}
\authornote{Both authors contributed equally to this research.}
\email{johndoe@johndoe.edu}

\begin{comment}

\author{Manish Motwani}
\email{manish.motwani@oregonstate.edu}

\end{comment}
\begin{comment}

\author{G.K.M. Tobin}
\authornotemark[1]
\email{webmaster@marysville-ohio.com}
\affiliation{%
  \institution{Oregon State University}
  \streetaddress{P.O. Box 1212}
  \city{Dublin}
  \state{Ohio}
  \country{USA}
  \postcode{43017-6221}
}

\author{Valerie B\'eranger}
\affiliation{%
  \institution{Inria Paris-Rocquencourt}
  \city{Rocquencourt}
  \country{France}
}


\author{Aparna Patel}
\affiliation{%
 \institution{Rajiv Gandhi University}
 \streetaddress{Rono-Hills}
 \city{Doimukh}
 \state{Arunachal Pradesh}
 \country{India}}

\author{Huifen Chan}
\affiliation{%
  \institution{Tsinghua University}
  \streetaddress{30 Shuangqing Rd}
  \city{Haidian Qu}
  \state{Beijing Shi}
  \country{China}}

    
\author{Charles Palmer}
\affiliation{%
  \institution{Palmer Research Laboratories}
  \streetaddress{8600 Datapoint Drive}
  \city{San Antonio}
  \state{Texas}
  \country{USA}
  \postcode{78229}}
\email{cpalmer@prl.com}

\author{John Smith}
\affiliation{%
  \institution{The Th{\o}rv{\"a}ld Group}
  \streetaddress{1 Th{\o}rv{\"a}ld Circle}
  \city{Hekla}
  \country{Iceland}}
\email{jsmith@affiliation.org}

\author{Julius P. Kumquat}
\affiliation{%
  \institution{The Kumquat Consortium}
  \city{New York}
  \country{USA}}
\email{jpkumquat@consortium.net}

%%
%% By default, the full list of authors will be used in the page
%% headers. Often, this list is too long, and will overlap
%% other information printed in the page headers. This command allows
%% the author to define a more concise list
%% of authors' names for this purpose.
\renewcommand{\shortauthors}{Trovato et al.}
\end{comment}
%%
%% The abstract is a short summary of the work to be presented in the
%% article.
\begin{abstract}
Recent research into improving FL has shown that taking multiple FL methods, preferably ones that use different kinds of information about the bug, and using machine learning to learn how to combine these methods’ rankings into a single ranking can significantly improve FL performance.  However, these learn-to-rank methods require a large labeled dataset of defects (very large for deep learning methods). In this paper, we ask the question whether this learning is necessary.  Can unsupervised methods – which require no training data – for combining ranked lists perform as well as these supervised methods?  We defevelop RAFL, borrowing the blah algorithms from the blah domain and show that it performs just as well as existing, state-of-the-art learn-to-rank methods without needing the large training data sets and the expensive training process.  Our results suggest that while ML may have useful applications in SE, there is evidence that combining FL methods is not one of them.

\end{abstract}

%%
%% The code below is generated by the tool at http://dl.acm.org/ccs.cfm.
%% Please copy and paste the code instead of the example below.
%%
%%
%% Keywords. The author(s) should pick words that accurately describe
%% the work being presented. Separate the keywords with commas.
\begin{comment}
\keywords{Do, Not, Us, This, Code, Put, the, Correct, Terms, for,
  Your, Paper}
%% A "teaser" image appears between the author and affiliation
%% information and the body of the document, and typically spans the
%% page.
\end{comment}

%%
%% This command processes the author and affiliation and title
%% information and builds the first part of the formatted document.
\maketitle

\section{Introduction}

Software practitioners invest a substantial amount of their time in the arduous process of debugging their code, primarily due to the time-consuming task of pinpointing faults within the software. Despite the existence of automated fault localization techniques for many years, the software industry has shown reluctance in embracing these methods, largely attributable to their suboptimal performance in accurately identifying and isolating faults.

Effective fault localization is crucial for software development, yet existing techniques often rely on bug reports or test suites, which may not be readily available in practical scenarios. Recent research has demonstrated that combining fault localization techniques with supervised machine learning algorithms can enhance performance. However, these supervised learning approaches face challenges in generalizability and require large labeled datasets for training.

Unsupervised techniques, exemplified by [RAFL], have shown superior fault localization performance compared to state-of-the-art supervised techniques. Nonetheless, their reliance on the availability of underlying fault localization technique results poses challenges, especially in real-world industry scenarios where such information may not be accessible.

This research addresses these challenges by proposing innovative technology aimed at automating the fault localization process, thereby increasing productivity for software practitioners. The goal is to reduce reliance on manual bug localization, which is time-consuming and diverts resources from potentially more impactful activities.


\section{Motivation}


\begin{comment}
    

\subsection{Template Styles}

The primary parameter given to the ``\verb|acmart|'' document class is
the {\itshape template style} which corresponds to the kind of publication
or SIG publishing the work. This parameter is enclosed in square
brackets and is a part of the {\verb|documentclass|} command:
\begin{verbatim}
  \documentclass[STYLE]{acmart}
\end{verbatim}

Journals use one of three template styles. All but three ACM journals
use the {\verb|acmsmall|} template style:
\begin{itemize}
\item {\texttt{acmsmall}}: The default journal template style.
\item {\texttt{acmlarge}}: Used by JOCCH and TAP.
\item {\texttt{acmtog}}: Used by TOG.
\end{itemize}

The majority of conference proceedings documentation will use the {\verb|acmconf|} template style.
\begin{itemize}
\item {\texttt{acmconf}}: The default proceedings template style.
\item{\texttt{sigchi}}: Used for SIGCHI conference articles.
\item{\texttt{sigplan}}: Used for SIGPLAN conference articles.
\end{itemize}

\subsection{Template Parameters}

In addition to specifying the {\itshape template style} to be used in
formatting your work, there are a number of {\itshape template parameters}
which modify some part of the applied template style. A complete list
of these parameters can be found in the {\itshape \LaTeX\ User's Guide.}

Frequently-used parameters, or combinations of parameters, include:
\begin{itemize}
\item {\texttt{anonymous,review}}: Suitable for a ``double-anonymous''
  conference submission. Anonymizes the work and includes line
  numbers. Use with the \texttt{\acmSubmissionID} command to print the
  submission's unique ID on each page of the work.
\item{\texttt{authorversion}}: Produces a version of the work suitable
  for posting by the author.
\item{\texttt{screen}}: Produces colored hyperlinks.
\end{itemize}

This document uses the following string as the first command in the
source file:
\begin{verbatim}
\documentclass[sigconf]{acmart}
\end{verbatim}
\end{comment}

\section{Implementation}

\section{Evaluation}

\section{Discussion and Threats to Validity}

\section{Related Work}
	\subsection{Improving Fault localization}
		
	Automated fault localization techniques leverage both static and run-time information of a program to identify potential program elements responsible for faults. SBFL computes a suspiciousness score for each program element by utilizing test coverage information whereas the MBFL techniques utilizes mutation analysis by analyzing the impact of artificially introduced program mutations on test results . On the other hand, Deep Learning-Based Fault Localization (DLFL) employs neural networks to create a fault localization model, utilizing the trace matrix and test results.
	
	The TRAIN \cite{HU2024111900} approach improves upon SBFL by identifying and excluding test cases that execute faulty statement(s) but lead to a correct output, thereby optimizing the trace matrix. Mutation-Spectrum Fault Localization \cite{dutta2021msfl} (MSFL) \cite{dutta2021msfl} combines mutation-based testing with SBFL. MSFL generates spectra for each mutant and the faulty program to combine them with SBFL techniques such as Tarantula, Barinel, Ochiai, and DStar to produce statement ranking sequences. Bug localization is then performed based on the similarity between the statement ranking sequence of the faulty program and mutants. However, MSFL assumes the thorough testing and fault-free nature of the original program thus it's effectiveness is compromised if no mutants are available for the faulty line, leading to a lower rank assigned to the faulty statement. WEGAT \cite{yanimproving}  represents the coverage matrix using a weighted execution graph by applying predicate weighted sequences combined with Abstract Syntax Tree (AST) information and feeds it to a Graph Attention Network for fault localization. However, the necessity for program instrumentation to collect predicate sequences incurs a time overhead, potentially hindering practitioner productivity and diverging from our goal of productivity enhancement.
	
	Recently, the potential of using LLMs for various code related tasks, especially fault localization, has been explored.LLMAO \cite{adian2023LLMAO} is a language model-based approach for fault localization that does not require test cases and locates buggy lines of code. It leverages large language models and bidirectional adapter layers to achieve high fault localization performance to detect general logic as well as security bugs in the code.
	Existing research has validated the effectiveness of integrating results from multiple fault localization (FL) techniques, revealing superior fault localization performance compared to standalone approaches. Noteworthy combination techniques, including CombineFL\cite{zou2019empirical}, DeepFL\cite{li2019deepfl} and Fluccs\cite{sohn2017fluccs} leverage learning to rank methodologies, notably RankSVM, to merge outputs from diverse FL techniques. However, these supervised approaches require labeled training datasets. The unavailability of such datasets, and even if one is willing to create them, introduces a substantial overhead, thereby contradicting the primary goal of enhanced practitioner productivity. Consequently, the practical application of these methodologies in real-world software industry settings is deemed unfeasible.

	Recent research introduces an unsupervised fault localization technique known as SBIR \cite{motwani2023better} to combine SBFL and Blues(IRFL) results to acheive better automatic program repair. SBIR utilises RAFL (Rank Aggregation for Fault Localization) technique that employs the Cross Entropy Monte Carlo Algorithm and the Spearman footrule distance to merge top-k ranked lists of suspicious statements and notably yields superior results when compared to the RankSVM approach employed by many state-of-the-art supervised models. This unsupervised technique addresses the challenges associated with labeled training datasets, making it a more viable solution for practical implementation in real-world software industry scenarios, however, the finetuning of parameters for RAFL to achieve optimal fault localization results for practitioners remains unexplored.
	
	
 

\section{Rights Information}

\section{Contributions}

\section{Acknowledgements}

\bibliographystyle{plain}

\bibliography{./references}


\end{document}
\endinput
%%
%% End of file `sample-sigconf.tex'.
